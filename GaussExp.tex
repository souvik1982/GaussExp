\documentclass[10pt,letterpaper]{article}
\usepackage{hyperref}
\usepackage{amsmath}
\usepackage{empheq}

\begin{document}

\title{A Poor Man's Crystal Ball}
\date{\today}
\author{Souvik Das\\ Department of Physics \\ University of Florida}
\maketitle

\begin{abstract}
A simpler alternative to the Crystal Ball function is presented here that has an exponential tail stitched to the Gaussian core. It has one parameter less than the Crystal Ball function, and where appropriate, offers more stable fits to peaks with exponential tails. The function, therefore, may be of general utility in experimental physics.
\end{abstract}

\bigskip

The Crystal Ball function, developed within the Crystal Ball Collaboration~\cite{Oreglia:1980cs, Skwarnicki:1986xj}, is a continuously differentiable ($C^1$) function that is often used as a probability density function in high energy physics. It is typically used to model lossy processes like the reconstructed invariant mass of a resonance from the energies and momenta of its decay products where some fraction of the energies and momenta are lost to detection. The Crystal Ball function consists of a power law tail stitched to a Gaussian core such that the function and its first derivative are continuous, as described in Eq.~\ref{eq:CrystalBall}. This results in 4 parameters: $\alpha$, $n$, $\bar{x}$ and $\sigma$. The power law parameter $n$ appears in the formula as $n^n$ and this sometimes makes the Crystall Ball an unstable fitting function. In this document, we report a simpler $C^1$ function that may be used to model similar peaks with energy losses that result in long tails. It consists of an exponential tail stitched to the Gaussian core such that the function and its first derivative are continuous.

\begin{eqnarray}
\label{eq:CrystalBall}
f(x; \alpha, n, \bar{x}, \sigma) &=& e^{-\frac{1}{2}\left({\frac{x-\bar{x}}{\sigma}}\right)^2} \quad \textrm{for} \quad \frac{x-\bar{x}}{\sigma} > -\alpha \\
                                 &=& \left(\frac{n}{|\alpha|}\right)^n e^{-\frac{|\alpha|^2}{2}} \left( \frac{n}{|\alpha|} - |\alpha| - \frac{x-\bar{x}}{\sigma} \right)^{-n} \quad \textrm{for} \quad \frac{x-\bar{x}}{\sigma} \leq -\alpha \nonumber
\end{eqnarray}

The new function is described in Eq.~\ref{eq:GaussExp} with the exponential tail on the lower side of the Gaussian. $\bar{x}$ and $\sigma$ represent the mean and standard deviation of the Gaussian core, respectively. $k$ represents the decay constant of the exponential tail. Requiring continuity of the function and its first derivative implies that $k$ is also the number of standard deviations on the side of the tail where the Gaussian switches to the exponential.

\begin{eqnarray}
\label{eq:GaussExp}
f(x; \bar{x}, \sigma, k) &=& e^{-\frac{1}{2}\left(\frac{x-\bar{x}}{\sigma}\right)^2}, \quad \textrm{for} \quad \frac{x-\bar{x}}{\sigma} \geq k  \\
                         &=& e^{\frac{k^2}{2}+k\left(\frac{x-\bar{x}}{\sigma}\right)}, \quad \textrm{for} \quad \frac{x-\bar{x}}{\sigma} < k \nonumber
\end{eqnarray}

A similar function with exponential tails on both sides of the Gaussian core is expressed in Eq.~\ref{eq:ExpGaussExp}. $k_L$ and $k_H$ are the decay constants of the exponentials on the low and high side tails.

\begin{eqnarray}
\label{eq:ExpGaussExp}
f(x; \bar{x}, \sigma, k_L, k_H) &=& e^{\frac{k_L^2}{2}+k_L\left(\frac{x-\bar{x}}{\sigma}\right)}, \quad \textrm{for} \quad \frac{x-\bar{x}}{\sigma} \leq k_L \\
                         &=& e^{-\frac{1}{2}\left(\frac{x-\bar{x}}{\sigma}\right)^2}, \quad \textrm{for} \quad k_L <  \frac{x-\bar{x}}{\sigma} \leq k_H  \nonumber \\
                         &=& e^{\frac{k_H^2}{2}-k_H\left(\frac{x-\bar{x}}{\sigma}\right)}, \quad \textrm{for} \quad k_H < \frac{x-\bar{x}}{\sigma} \nonumber
\end{eqnarray}

A recent CMS paper on the search for resonant pair production of Higgs bosons decaying to two bottom quark-antiquark pairs~\cite{Khachatryan2015560} has used these functions to model the invariant masses of signal and background processes. In the paper they are referred to as ``GaussExp" and the ``ExpGaussExp" functions, respectively. These functions may find more general use in experimental physics.

\bibliographystyle{plain}
\bibliography{GaussExp}

\end{document}
